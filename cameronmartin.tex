\documentclass{scrartcl}


\usepackage[english]{babel} 

\usepackage{amsfonts}
\usepackage{amsmath,amsthm,amssymb,upref}
\usepackage[left=1cm, right = 1cm, top = 1cm, bottom = 1cm]{geometry}


\newtheorem{lemma}{Lemma}
\theoremstyle{definition}
\newtheorem{definition}{Definition}
\theoremstyle{remark}
\newtheorem{remark}{Remark}

\newcommand{\Frechet}{Fr\'echet }
\newcommand{\T}{\mathcal T}
\newcommand{\eps}{\varepsilon}
\newcommand{\N}{\mathbb N}
\newcommand{\R}{\mathbb R}
\newcommand{\Q}{\mathbb Q}
\newcommand{\E}{\mathbb E}

\newcommand{\del}{\partial}
\newcommand{\dist}{\operatorname{dist}}

\begin{document}
\section{\Frechet spaces}
Let $X$ be a topological vector space with topology $\T$. A few preliminary definitions:
\begin{definition}
A set $C$ is called
\begin{itemize}
\item balanced if for all $x\in C$ and all $|\lambda| \leq 1$ also $\lambda x \in C$.
\item convex if for all $x, y\in C$ and all $0\leq t \leq 1$, also $tx+(1-t)y\in C$.
\item absorbent if $\bigcup_{t>0} tC = X$.
\item absolutely convex, if $C$ is both convex and balanced. This is the same as saying that for every $x,y\in C$ and real numbers $|t|+|s|=1$, $tx+sy \in C$. The best image for this is that $C$ contains every parallelogramm with vertices $x$ and $y$ symmetric to the origin.
\end{itemize} 
\end{definition}
\begin{definition}
\begin{itemize}
\item A collection of sets $\mathcal B \subset \T$ is called a base of the topology $\T$ if for every open set $U\in\T$ there is a $B\in \mathcal B$ with $B\in U$.
\item A collection of sets $\mathcal B_x \subset \T$ is called a local base at $x$ of the topology $\T$ if for every neighborhood $U\in\T$ of $x$ (i.e. $x\in U$) there is a $B\in \mathcal{ B}$ with $B\in U$.
\end{itemize}
\end{definition}

We are going to compare two definitions of locally convex spaces.

\begin{minipage}[t]{0.45\textwidth}
\begin{definition}\label{def:l1}
A topogical vector space $X$ is called \textit{locally convex} if the origin has a local base of absolutely convex absorbent sets.
\end{definition}
\end{minipage}
\hfill
\begin{minipage}[t]{0.45\textwidth}
\begin{definition}\label{def:l2}
A topological vector space $X$ is called \textit{locally convex} if there is a family of seminorms generating the topology.
\end{definition}
\end{minipage}

\begin{lemma}
Definitions \ref{def:l1} and \ref{def:l2} are equivalent.
\end{lemma}
\begin{proof}
{\textit{Proof of \ref{def:l1} $\Rightarrow$ \ref{def:l2}.}}
Define the Minkowski gauge of a set $C\in X$ as 
\[\mu_C(x) = \inf\{\lambda> 0: x\in \lambda C\}.\]
Then we can show that the origin's local base of absolutely convex absorbent sets $\mathcal{B}_0 = \{B_a, a\in I\}$ constitute a family of seminorms via their Minkowski gauges $\{\mu_{B_a}, a\in I\}$ and the latter generates the topology of $X$. XXX

{\textit{Proof of \ref{def:l2} $\Rightarrow$ \ref{def:l1}}}
Using the family of seminorms we can easily build a local base of absolutely convex absorbent sets for the origin.XXX
\end{proof}

\begin{remark}\label{rem:locconv}

\begin{enumerate}
\item In Definition \ref{def:l2}, note that the family of seminorms is not necessarily countable.
\item If the family of seminorms $\|\cdot\|_k$ generating the topology is \textit{countable}, then this constitutes a pseudometric generating the topology. To see this, define
\begin{equation} d(x,y) = \sum_{k=1}^\infty 2^{-k} \frac{\|x-y\|_k}{1+\|x-y\|_k}.\label{eq:metric}
\end{equation}
Note that this pseudo	metric is neither unique (so the metrization is not unique) nor homogenous (so we can't define a suitable norm). 
\item As translation is continuous in any t.v.s, we can use a base for the origin as a base for any point by translation.
\end{enumerate}
\end{remark}

\begin{lemma}\label{lem:hausdorff}
Let $X$ be a locally convex TVS with its family of seminorms $(\| \cdot \|_\alpha)_{\alpha\in I}$. Then $X$ is Hausdorff if and only if
\begin{equation}\label{eq:hausdorff}
\forall \alpha\in I:~ \|x\|_\alpha = 0\quad \Leftrightarrow \quad x = 0
\end{equation}
\end{lemma}
\begin{proof}
Let $X$ be Hausdorff. We show \eqref{eq:hausdorff}. If for every $\alpha\in I$, $\|x\|_{\alpha}=0$ and we assume that $x\neq 0$, we know by the Hausdorff property that there exists an open neighborhood $U_x$ of $x$ and another open neighborhood of $0$ such that they do not intersect. As the topology is generated by the seminorms, we can set $U_x = \{u:~ \|u-x\|_{\alpha} < \eps~ \forall k\leq K\}$ for some $K\in \N, \eps>0$. Now obviously $0\in U_x$ as $\|x\|_{\alpha} = 0$ for all $\alpha$. Hence there cannot exist any neighborhood of $0$ such that we can separate $x$ and $0$. The other direction of \eqref{eq:hausdorff} is obvious by the definition of seminorms.

Now suppose \eqref{eq:hausdorff} holds. If we set $x\neq y$, by \eqref{eq:hausdorff} we know that there is at least one $\alpha\in I$ such that $\|x-y\|_\alpha = M \neq 0$.

By setting $U_x =\{u\in X:~\|u-x\|_\alpha < M/2 \}$ and $V_y =\{v\in X:~\|v-y\|_\alpha < M/2 \}$, we have $U_x\cap V_y = \emptyset$ by the triangle inequality (assume there is a $w$ both in $U_x$ and $V_y$, then one can show that $\|x-y\|_\alpha \leq \|x-w\|_\alpha + \|w-y\|_\alpha < M$).
\end{proof}

We are going to compare two definitions of Fr\'echet spaces.\newline

\begin{minipage}[t]{0.45\textwidth}
\begin{definition}\label{def:f1}
A \Frechet space $X$ is a topological vector space $X$ such that
\begin{enumerate}
\item $X$ is locally convex, i.e. $0\in X$ has a local base of absorbent and absolutely convex sets.
\item The topology $\T$ can be induced by a translation invariant metric $d$, i.e. $d(x+a, y+a) = d(x, y)$.
\item $(X, d)$ is a complete metric space.
\end{enumerate}
\end{definition}
\end{minipage}
\hfill
\begin{minipage}[t]{0.45\textwidth}
\begin{definition}\label{def:f2}
A \Frechet space $X$ is a topological vector space $X$ such that
\begin{enumerate}
\item The topology can be induced by a countable family of seminorms $\|\cdot \|_k$, i.e. $U$ open if and only if for all $u\in U$ there is an integer $K \geq 0$ and an $\eps > 0$ such that \[\{v: \|v-u\|_k<\eps:~ k\leq K\} \subset U\]
\item $X$ is Hausdorff.
\item $X$ is complete w.r.t. the family of seminorms $\|\cdot\|_k$, i.e. if $(x_m)_m$ is a Cauchy sequence w.r.t. all seminorms, then there exists an $x\in X$ with $x_n\to x$ w.r.t. $\|\cdot \|_k$ (and by property 2 even $x_n\to x$).
\end{enumerate}
\end{definition}
\end{minipage}

\begin{remark}
\begin{enumerate}
\item Note that the items 1-3 are not all ``parallely equivalent", but as a set, they are equivalent:
\begin{itemize}
\item \ref{def:f1}.1 and \ref{def:f1}.2 $\Rightarrow$ \ref{def:f2}.1 (the countability property of the family of seminorms is stronger than plain local convexity)
\item \ref{def:f1}.2 $\Rightarrow$ \ref{def:f2}.2 (metrizability implies Hausdorff)
\item \ref{def:f2}.1 and \ref{def:f2}.2 $\Rightarrow$ \ref{def:f1}.2
\item \ref{def:f2}.1$\Rightarrow$ \ref{def:f1}.1 
\item \ref{def:f1}.3 $\Leftrightarrow$ \ref{def:f2}.3
\end{itemize}
\end{enumerate}
\end{remark}
\begin{lemma}
Definitions \ref{def:f1} and \ref{def:f2} are equivalent.
\end{lemma}
\begin{proof}
\textbf{\ref{def:f1} $\Rightarrow$ \ref{def:f2}.}

ad 1.): From the equivalent definitions of locally convex vector spaces we know that \ref{def:f2}.1 is \textit{almost} local convexity. The trouble is that we claim the existence of a \textit{countable} family of seminorms whereas local convexity just gives some family of seminorms. By combining \ref{def:f1}.1 (local convexity in its set-theoretic version) and \ref{def:f1}.2 (metrizability), we obtain \ref{def:f2}.1.

Indeed, the metric $d$ allows us to define small balls $B_n = \{x:~ d(x,0)< \frac{1}{n}\}$ for each $n\in\N$. As the topology can be induced by $d$, we know that there is an open set inside each $B_n$ and by local convexity we know of the existence of an absorbent and absolutely convex set $C_n \subset B_n$. Those sets in turn define via their Minkowski gauges $\mu_n$ a countable family of seminorms. Equivalence of the topology of $X$ and the topology generated by the family $\mu_n$ follows from the fact that the $C_n$ are a local base of the topology of $X$.

ad 2.): Every metric (or metrizable) space is Hausdorff.

ad 3.): This is a direct consequence of completeness of $d$.

\textbf{\ref{def:f2} $\Rightarrow$ \ref{def:f1}}
ad 1.): As noted above, \ref{def:f2}.1 is slightly stronger than \ref{def:f1}.1.

ad 2.): By \ref{def:f2}.2 we know by lemma \ref{lem:hausdorff} that the countable family of seminorms in \ref{def:f2}.1 fulfills
\[\forall n\in \N:~ \|x\|_k = 0 \quad \Leftrightarrow \quad x = 0.  \] 
This property makes the pseudometric in \eqref{eq:metric} (which is one possible pseudometrization) a proper metric. The fact that the topology of $X$ is induced by $d$ is equivalent to \ref{def:f2}.1.
\end{proof}
\section{The space $\R^\infty$}
\begin{lemma}
The topological vector space $W = \R^\infty = \{(x_n)_{n\in\N}: ~x_n\in\R:~\forall n\in\N \}$ with the product topology (i.e. the topology generated by all sets of the form $U_1\times\cdots\times U_m\times \R^\infty$ with all $U_i\subset \R $ open) is a \Frechet space. One possible metrization is
\begin{equation}
d(x, y) = \sum_{n=1}^{\infty}2^{-n}\cdot \frac{|x_n-y_n|}{1 + |x_n-y_n|}
\end{equation}
\end{lemma}
\begin{proof}
We will prove this by showing that $\R^\infty$ fulfills every item in definition \ref{def:f2}.

ad 1.): We define $\|x\|_k := |x_k|$, i.e. the absolute value of the $k$-th item. This is obviously a seminorm on $\R^\infty$. We show that the product topology an be generated by this family of seminorms. 

Let $U'$ be an open set in the product topology. Then $U'$ is a union of sets of the form $U_1\times\cdots\times U_m\times \R^\infty$. It suffices to consider only open sets out of this basis. Let $U = U_1\times\cdots\times U_m\times \R^\infty$. Choose a point $u\in U$, i.e. $u = (u_1, \ldots, u_m, u_{m+1},\ldots)$ where $u_i \in U_i$ for $i=1,\ldots, m$. Define $\delta = \min_{i=1}^M \dist(u_i, \del U_i)$, where $\dist$ is the distance function on $\R$. Then the set $\{v: \|v-u\|_k < \delta: k\leq M\} $ is a subset of $U$, which constitutes the first part of Definition \ref{def:f2}.1. The other direction is obvious by definition of the product topology.

ad 2.): By lemma \ref{lem:hausdorff} we know that we only need to show that if and only if for given $u\in\R^\infty$ and all $n\in \N$ we have $\|u\|_k = |u_k| = 0$, then $u=0\in\R^\infty$. But this is obvious.

ad 3.): This is easily shown by using completeness of $\R$ in each dimension of $\R^\infty$.
\end{proof}

\begin{lemma}
The space $\R^\infty$ with the product topology is a polish space.
\end{lemma}
\begin{proof}
$\R^\infty$ is separable as $\Q^\infty$ is countable and dense and it is completely metrizable by definition of \Frechet spaces.
\end{proof}

\begin{lemma}
The product topology of $\R^\infty$ cannot be generated by a norm. 
\end{lemma}
\begin{proof}
Open sets of $\R^\infty$ are necessarily unbounded but balls defined by any norm must be (by definition) bounded.
\end{proof}
\begin{lemma}
The Borel $\sigma$-algebra is the same as the product $\sigma$-algebra.
\end{lemma}
\begin{proof}
The Borel-$\sigma$-algebra is the $\sigma$-algebra generated by open sets (i.e. sets of the form $U_1\times\cdots\times U_m\times \R^\infty$), which is the same as the product $sigma$-algebra.
\end{proof}
We write $e_i$ for the element of $W$ with $e_i(j) = \delta_{ij}$. Any element $x\in W$ can be written as $x = \sum_{i=1}^\infty x_i e_i$.
\begin{lemma}
Every continuous linear functional $f\in W^*$ is of the form 
\begin{equation}f(x) = \sum_{i=1}^n a_i x_i \label{eq:darstellung_f} \end{equation}
for some $a_1,\ldots, a_n\in\R$ and with the notation $W \ni x = (x_1, x_2,\ldots)$. Thus $W^*$ can be identified with $c_{00}$, the set of all real sequences which are eventually zero.
\end{lemma}
\begin{proof}
Let $f\in W^*$. Define $a_i = f(e_i)$. We show that only finitely many $a_i$ are nonzero. Indeed, assume that for any $N\in \N$ there is an $i_N > N$ such that $a_{i_N} \neq 0$.

As we assumed $f$ to be continuous, for any $\eps > 0$ there is an open neighborhood $U$ of $0$ such that

 \begin{equation}
\sup_{u\in U} |f(u)| < \eps.  \label{lem:cont}
 \end{equation} 
 By the form of the topology, this $U$ is of the form
\[ U =  \bigotimes_{i=1}^n U_i \times \R^\infty \]
for some $n\in \N$. Now we can define a sequence $(u_N)_N$ of elements $u_N$ (which are sequences) in $W$ as $u_N = \frac{1}{a_{i_N}}\cdot N\cdot e_{i_N }$, i.e. the $N$-th element is the sequence which has the value $\frac{N}{a_{i_N}}$ at the $i_N$-th position and $0$ elsewhere. For $M$ high enough, $(u_N)_{N\geq M}$ is a sequence in $U$. But
\[ f(u_N) = \frac{N}{a_{i_N}}\cdot f(e_{i_N}) = N \xrightarrow{N\to\infty} \infty,\]
hence \eqref{lem:cont} is violated and $f$ cannot be continuous.

This means that $f$ acts on only a finite number of basis elements of $W$ and as $f$ is linear we can write it in the form \eqref{eq:darstellung_f}.
\end{proof}
We choose the measure $\mu$ to be an infinite product of Gaussian measures with variance $1$ and mean $0$. So the projections of elements $x\in W$ on the coordinates are i.i.d. Gaussians.
\begin{lemma}\label{lem:Gaussian}
$\mu$ is a Gaussian measure. The covariance form of $\mu$ is given by
\[ q(f,g) = \sum_{i=1}^\infty f(e_i)g(e_i) \]
where the sum is actually finite.
\end{lemma}
\begin{proof}
We argue by using characteristic functions. 
\begin{align*}
\int_W e^{if(x)}\mu(dx) &= \int_{\R^\infty} e^{i\sum_{j=1}^n f(e_j) x_j}\, \bigotimes_{j=1}^\infty N(0,1)(dx_j) = \prod_{j=1}^n \int_\R e^{if(e_j)x_j}N(0,1)(dx_j)\\
&= \prod_{j=1}^n e^{-\frac{f(e_j)^2}{2}} = e^{-\frac{1}{2}\sum_{j=1}^n f(e_j)^2}
\end{align*}
and thus the covariance form is $q(f, f) = \sum_{j=1}^n f(e_j)^2$ and hence $q(f,g)=\sum_{j=1}^\infty f(e_j)g(e_j)$ where the sum runs until the largest non-zero entry of both $f$ and $g$.
\end{proof}
\begin{lemma}
$\mu$ has full support, i.e. there is no open set other than the empty set having zero measure.
\end{lemma}
\begin{proof}
Let $U$ be open in $W$, i.e. $U = \bigotimes_{i=1}^n U_i \times \R^\infty$ with $U_j$ open in $\R$. Then $\mu(U) = \prod_{j=1}^n N(0,1)(U_j) > 0$ as $N(0,1)$ has full support on $\R$.
\end{proof}
\begin{remark}
$q$ is actually positive definite: The only $f\in W^*$ with $q(f, f) = 0$ is $f = 0$. This means that $i: W^*\hookrightarrow L^2(W, \mu)$ with inner product $q$ is an injection in the sense that $i(W^*) \subset L^2$ is isomorphic to $W^*$. This is actually cool because normally there may be $0 \neq f \in W^*$ with $0 = f \in L^2$ (for example in spaces with degenerate Gaussians like $W= \R^2$ with $N(0,1) \otimes \delta_0$. Here, choosing $f$ as the projection on the second coordinate (written as $f = (0,1)^T$) is not $0$ as a linear functional in $W^* = \R^2$ but $f(x) = 0$ $\mu$-almost surely and thus $0 = f\in L^2$ as a square-integrable random variable.
So, to reiterate, $W^*\hookrightarrow L^2(W, \mu)$ is an injection and we can view $W^*$ as a \textbf{subspace} of $L^2$. This is because we don't lose ``information'' by viewing $f$ as an element in $L^2$. In the example above with $\R^2$ we do lose information because after identifying $W^*$ with a subset of $L^2$, we can't distinguish $(0,1)^T$ and $(0,0)^T$ as random variables in $L^2$, although they are quite distinct in $W^* = \R^2$.
\end{remark}

\begin{remark}
As was pointed out, we can think of $W^*$ as a subset of $L^2(W,\mu)$ with inner product $q$. But: $W^*$ is not complete in the $q$ inner product. Let's define $K = \overline{W^*}^{L^2(W,\mu)}$, the $L^2$ -closure of $W^*$.
\end{remark}
\begin{lemma}
$K$ consists of all functions $f:W\to \R$ of the form
\begin{equation}\label{eq:l2}
f(x) = \sum_{j=1}^\infty a_j x_j
\end{equation}
with $\sum_{j=1}^\infty |a_j|^2 < \infty$.
\end{lemma}
\begin{proof}
As $W^*$ can be identified with $c_{00}$ and $L^2(W,\mu$ with inner product $q$ can be identified with $l^2$, the space of square summable sequences, this is equivalent to $\overline{c_{00}}^{l^2} = l^2$, which is standard.
\end{proof}
\begin{remark}
Note one subtle point about \eqref{eq:l2}: For an arbitrary $x\in W$, the sum may not converge. It does, however, converge for $\mu$-a.e. $x\in W$. Indeed: Sums of independent random variables (and the $a_j\cdot x_j$ are i.i.d Gaussians) converge a.s. as soon as the sum of their variances converge. Thus, we only need to show
\[ \sum_{j=1}^\infty a_j^2 \cdot \E x_j^2 < \infty \]
which follows immediately by $\E x_j^2 = 1$ and the condition on $(a_j)_j$.

By the same calculation as in lemma \ref{lem:Gaussian}, we can show that $f$ is a Gaussian random variable with covariance form $q(f, g) = \sum_{j=1}^\infty f(e_j)g(e_j)$ where now the sum may contain infinitely many terms (but still is of finite value by the Cauchy-Schwarz inequality in $l^2$).
\end{remark}
\end{document}
